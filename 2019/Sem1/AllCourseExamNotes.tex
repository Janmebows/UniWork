\documentclass{X:/Documents/Coding/Latex/myassignment}
\title{Exam Notes 2019}
\begin{document}

\section{Diseases (Topic B)}
Exam 5th July - EMG07 from 9:20am

Exam:
\begin{itemize}
    \item 8 questions
    \item 92 marks
    \item 3 hour exam
\end{itemize}
Questions

\subsection{10 true false questions with brief justification (20 marks)}

Can't put much here
\subsection{ODE - SIR model (5 marks)}
a
\subsection{Characteristics of different model types (reasoning) (6 marks)}
a
\subsection{Specifying a CTMC (interpreting from words) and simulation (14 marks)}
For a CTMC you must give a state space, $x(t) \in S \forall t\geq 0$, and state transitions (or a generator). 
\subsection{CTMC model and deterministic approximation (11 marks)}

\subsection{Branching processes (14 marks)}

\subsection{Path integrals (12 marks)}

\subsection{Bayesian inference (10 marks)}



$SI$ model
\[\odd It = \begin{cases}
	\frac{\beta I (N-I)}{N}, &\text{ FDT }\\
	\beta I (N-I), &\text{ DDT }\\	
\end{cases}\]

Non dimensionalisation by letting $i=\frac{I}{N}$, and same for $s$, $r$.


Analytic solution to the ODE
\[i(t) = \frac{i_0}{i_0 + (1-i_0)e^{-\beta t}}\]

Final size
\[r_\infty = 1 + \frac1{R_0} \mathcal{W}(-s_0e^{-R_0}R_0)\]
where $\mathcal{W}$ is the lambert W function (sol to $f(w) = we^{w}$)


Let $T_1$ be the time the outbreak ends
\[T_1 = \inf\{t|i(t) > 1-\frac1{N}\}\]
so sub $T_1$ into $i(t) = 1-1/N$
\[i(t) = \frac{i_0}{i_0+(1-i_0)e^{-\beta t}}\]

$P = \left(P_{ij}(t), \ i,j \in s, \ t\geq 0\right)$ is a transition function if
\begin{itemize}
	\item $P_{ij}(t)\geq 0$
	\item $P_{ij}(0) = \delta_{ij}$
	\item $\sum_{j\in S} P_{ij}(t) \leq 1 $
	\item $P_{ij}(t+s) = \sum_{k\in S} P_{ik}(s)P_{kj}(t)$
\end{itemize}

If $P$ is a standard transition function, i.e. $\lim_{t\downarrow 0} P_{ij}(t) = \delta_{ij}$
\begin{align*}
	q_i &= \lim_{t\downarrow 0} \frac{1-P_{ii}(t)}{t}\\
	q_{ij} &= \lim_{t\downarrow 0} \frac{P_{ij}(t)}{t}
\end{align*}
With $0\leq q_i \leq \infty$ and $0\leq q_{ij} < \infty$

Chapman-Kolmogorov
\[P_{ij}(s+t) = \sum_{k\in S} P_{ik}(s) P_{kj}(t), \quad s,t\geq 0\]

Kolmogorov-Forward eq
\[P'(t) = P(t) Q\]
\[P(t) = P(0)e^{Qt}\]

A state is recurrent if
\[\int_0^\infty P_{jj}(t) dt = \infty\]
Transient if $<\infty$.

It is positive recurrent if the mean return time $T_j$ is finite, null recurrent otherwise.


If $S_n$ is the time of the $n^{th}$ jump, with $S = \lim_{n\to\infty} S_n$
\[S_n = \sum_{i=1}^n T_i\]
If $E(S) < \infty$ then the chain performs an infinite $\#$ of jumps in a finite time with probability 1 (the chain explodes), and is not regular.

\[p(s+\tau)[I-\tau Q] \approx p(s)\]


A family of MCs is density dependent if 
\[q_{k,k+l} = rf(\frac{K}{r}, l), \quad l\neq 0\]
SIS is density dependent with
\[f(i,l) = \begin{cases}
	\beta(1-i)i,& l=1\\
	\gamma i,& l=-1
\end{cases}\]



Don't forget expectation
\[E(I(t)) = \sum_{I=0}^\infty IP_I(t)\]



Branching processes:

If the lifetime of an individual is exponentially distributed with rate $\mu$.
At the time of death an individual generates a random number of children with pmf
\[\{P_k\}_{k\geq 0}\]
and pgf
\[P(s) = \sum_{k=0}^\infty P_ks^k\]
So
\[P'(s) = \sum_{k=0}^\infty P_k s^{k-1}\]
so $P'(1) = \mathbb{E}(y) = m$, i.e. the mean number of offspring for one person

\[F(s,t) = \sum_{k\geq 0} P(X(t) = k) s^k\]
Where $F(s,t)$ is the p.g.f of $X(t)$ (the population size at $t$).

The first person is alive with probability $e^{-\mu t}$, since $X(t) =1$, $F(s,t)=s$

The first person will die in $(u,u+du)$ with probability $\mu e^{-\mu u} du$, and will have $N$ offspring with prob $P_N$. And so the number of people at time $t$ is
\[X(t) = \sum_{i=1}^N X_i(t-u)\]
Where $X_i(t-u)$ is the size of the subprocess generated by the $i^{th}$ child after $t-u$ units of time.
So the p.g.f of $X_i(t-u)$ is $F(s,t-u)$ and each is iid so the pgf of $X(t)$ is $F(s,t-u)^N$. And since $N$ is a random var
\[\sum_{k\geq 0}P_k F(s,t-u)^k = P(F(s,t-u))\]

Eventually get
\[\dd{F(s,t)}{t} = \mu( P(F(s,t))- F(s,t)) \]
since it is a PGF the expected pop at time $t$ diff wrt $s$ and then set $s=1$ (use $F(1,t) =1$)
Giving the mean population size, where $m=P'(1)$ i.e. the mean num of offspring
\[M(t) = e^{\mu (m-1)t}\]
\begin{itemize}
	\item $m>1$ gives $\lim_{t\to\infty} M(t) =\infty$ (outbreak)
	\item $m=1$, $M(t)=1$ for all $t$ (almost sure extinction)
	\item $m<1$ with $\lim_{t\to\infty} M(t) =0$ subcritical
\end{itemize}

If $q$ is the probability of extinction, $q$ is the minimal, non-negative solution to 
\[q = P(q)\]
Where $P(q) =\sum_{k\geq 0} p_k q^k$
In a normal branching process in the state 1 $P(q) = P_0 + P_2 q^2 = \frac{\gamma}{\gamma+\beta} + \frac{\beta}{\gamma+\beta} q^2$


This approximation is good for a minor outbreak for large $N$ or a major outbreak until $\sqrt{N}$ people are infected.


Household model 2 groups (internal house, external) $M$ houses with $N$ people in each house. Total pop $MN$ with SIR dynamics
The state space is $\vec m$ where $m_{S,I}$ corresponds to the number of hh's in each possible config. I.e. $m_{(i,j)}$ is the number of households with $i$ susceptible and $j$ infected
\[\vec m = \left(m_{(N,0)}, m_{(N-1,1)},\ldots, m_{(0,0)}\right)\]
3 events -
\begin{itemize}
	\item internal hh infection, $ \frac{\beta si}{N-1} m_{(s,i)} $
	\item external hh infection, $ \frac{\alpha si}{MN} m_{s,i}$
	\item recovery, $\left(m_{(s,i)}, m_{(s,i-1)} \right) \to \left(m_{(s,i)}-1,m_{(s,i-1)}+1)\right)$
\end{itemize}


Want to find $d_i$, the expected amount accumulated given the process started in state $i$. 
\[Q_B \vec d = - \vec f\]
Where $Q_B$ is the $Q$ matrix restricted to the non-absorbing states, and $f_i$ is the cost per unit time in state $i$, and $\vec f = \vec 1$ gives the expected time until absorption (extinction)


Branching process - 
If $X(t)$ is the population size at time $t$, with state space $S=\mathbb{N}$ and $0$ is absorbing
\[F(s,t) = \sum_{k\geq 0} P(X(t)=k)s^k\]




\[P(A|B) = \frac{P(B|A) P(A)}{P(B)}\]







\[(Q_B - lF)L = -a\]
Where $Q_B$ as usual
\[L = (L_i, i\in B)\]
\[F = \begin{pmatrix}
	f(1)\\
	&f(2)\\
	&&\ldots\\
	&&&f(|B|)
\end{pmatrix}\]







\end{document}