\documentclass{X:/Documents/Coding/Latex/myassignment}

\title{Modelling With ODEs Tutorials}

\begin{document}
\maketitle

\section{Tute 1}
\begin{enumerate}
    \item Fishery Model where $N(t)$ is the number of fish
    \[\frac{dN}{dt} = f(N) = BN - DN^2 - Y\]
Y fishing yield, B birth rate, D death rate.
In lectures we showed that a non-dimensional version of the model is
\[\frac{d\hat{N}}{d\hat{t}} = \hat{N}(1-\hat{N}) - y\]
\begin{enumerate}
    
    \item Show that the steady state is: 
    \[\hat{N}_*^{\pm} = \frac{1\pm \sqrt{1-4y}}{2}\]
    and which values of $y$ does it exist:
    Steady state when 
    \begin{align*}
        \hat{N}(1-\hat{N}) -y &= 0\\
        \hat{N} - \hat{N}^2 -y &= 0\\
        \hat{N}^2 - \hat{N} + y &= 0\\
        \hat{N} &= \frac{1 \pm \sqrt{1 - 4y}}{2}\\
        \frac{-b \pm \sqrt{b^2 - 4ac}}{2a}
    \end{align*}
    Exists if $\Delta \geq 0$, i.e. $1-4y \geq 0$.
    \[\implies 4y \leq 1 \implies y \leq \frac14\]
    \item Stability of the steady states?
    \begin{align*}
        f'(\hat{N}) &= 1 - 2\hat{N}\\
        \implies f'(\hat{N}^*_+) < 0 \\
        \implies f'(\hat{N}^*_-) > 0 
    \end{align*}
    So $\hat{N}^*_+$ is stable, and $\hat{N}^*_-$ is unstable. When $\hat{N}(0) < \hat{N}^*_-$ the population will go to $0$.
    \item When $y=.25$ we get a repeated root $\hat{N}^* = \frac12$.
    \[f'(N^*) = 0, \quad f''(N^*) = -2\]
    So since the slope is $0$ and it is a turning point, it is semi-stable (if we shift to the right it will come back, to the left it will continue to the left)
    
    \item What happens when $y > 0.25$? There are no real roots so there are no steady states...
    
\end{enumerate}

    
    \item 
    \begin{enumerate}
        \item Fixed points are where $C(x)$, $S(x:\mu)$ intersect, treat $S(x)$ as $x$, so rotate the coordinate system.
        \item 
        \begin{enumerate}
            \item Bifurcation points when
            \[f(\bar x:\bar\mu) = 0, \dd fx = 0\]
            \begin{align*}
                \mu x + x^3 - x^5 = 0\\
                x(\mu + x^2 - x^4) = 0\\
                x= 0 \ or \ x^4 - x^2 -\mu=0 \\
                \implies x^2 = \frac{1\pm\sqrt{1+4\mu}}{2}\\
                \implies x = \pm \sqrt{\frac{1\pm\sqrt{1+4\mu}}{2}}
            \end{align*}
            The four will exist/cease to exist for values of $\mu$ need both square roots to exist
            \[x_{\pm+} \implies \mu > -1/4\]
            \[x_{\pm-} \implies \mu \in \{-1/4,1/4\}\]
            I.e. bifurcation at $(x,\mu) = (1/2,-1/4), (-1/2,-1/4)$
            And $\bar{x}=0$ works for all $\mu$.
            % Which are valid given: $-\frac14<\mu$, and $\sqrt{1+ 4\mu} <1$
            % \begin{align*}
            %     \dd fx = \mu + 3x^2 - 5x^4\\
            %     0=\dd fx|_{\bar{x}} = \mu + 3 \frac{1\pm\sqrt{1+4\mu}}{2} - 5 \left(\frac{1\pm\sqrt{1+4\mu}}{2}\right)^2\\
            %     = \mu + \frac32 (1\pm\sqrt{1+4\mu})- \frac54 \left(1\pm\sqrt{1+4\mu}\right)^2\\
            %     = \mu + \frac32 (1\pm\sqrt{1+4\mu})- \frac54 \left(1\pm 2\sqrt{1+4\mu} + 1+4\mu\right)\\
            %     = \mu-5\mu  + \frac{3-5}2 \pm \sqrt{1+4\mu}(\frac32 - \frac52) \\
            %     = -4\mu  -2 \pm 2\sqrt{1+4\mu} \\
            %     0= -2\mu  - 1 \pm \sqrt{1+4\mu} \\
            %     (2\mu +1)^2=\pm \sqrt{1+4\mu}^2\\
            %     4\mu^2 + 4\mu + 1 = 1+4\mu\\
            %     4\mu^2  = 0\\
            %     \mu = 0
            % \end{align*}
            % \[\bar{x} = \pm \sqrt{\frac{1 \pm \sqrt{1}}{2}} = \pm \sqrt{0.5\pm0.5} = \pm 0, \pm 1\]
            % So the bifurcation occurs at
            % \[(\bar{x},\bar{\mu})\]
            % Gives
            \item This was found at the start:
            \[x = \pm \sqrt{\frac{1\pm\sqrt{1+4\mu}}{2}}, 0\]
            \item 
            \item 
        \end{enumerate}
        \item 
        \begin{enumerate}
            \item 
            \item 
        \end{enumerate}
    \end{enumerate}
    
    
\end{enumerate}




\section{Tute 3}
\begin{enumerate}
    \item Trajectories of form
    \[\mathbf{x} = e^{\alpha t}\begin{pmatrix}
        \cos \beta t\\ -\sin \beta t

    \end{pmatrix}\]
    $\alpha= 0$ gives centres and $\alpha\neq 0$ gives spirals. 
    \begin{enumerate}
        \item Effect of $\beta$ on the direction of trajectory: For $0\leq t \leq \beta 2\pi $
        $-\sin \beta t > 0$ when $\beta t \in (\pi, 2 \pi)$
        $\cos \beta t > 0$ when $\beta t \in (-\pi/2,\pi/2)$
        \item 
    \end{enumerate}
    \item Done in matlab
    \item 
    \[\frac{dx}{dt} = 3x - x^2 - xy, \quad \frac{dy}{dt} = 2y - y^2 - xy\]
    \begin{enumerate}
        \item Competition model
        \item $n_x = x=0, x = 3-y$, \\
        $n_y = y = 0, y = 2-x$

        Biologically relevant where $x,y \geq 0 $, steady states are:\\
        $x,y=0$\\
        $x = 3, y = 0$,
        $x = 0, y = 2$
        \item Linearisation
        \[J(x) = \begin{pmatrix}
            3 - 2x -y & -x\\ -y & 2-x - 2y
        \end{pmatrix}\]
At the steady states: 
\[J(0,0) = \begin{pmatrix}
    3&0\\0&2
\end{pmatrix} \implies \lambda_{1,2} > 0\implies unstable\]
\[J(3,0) = \begin{pmatrix}
    -3&-3\\ 0 & -1
\end{pmatrix} \implies \lambda = -1,-3 \implies  asymptotically\ stable\]
$\det(J(3,0)) = 3$ $trace(J(3,0)) = -4$
$\frac14 tr(J)^2 = 4$ stable node.



\[J(0,2) = \begin{pmatrix}
    1&0\\-2&-2
\end{pmatrix}\]



        \item n ty
        \item 
    \end{enumerate}


\end{enumerate}

\clearpage
\section{Tute 4}
\begin{enumerate}
    \item 
    \begin{enumerate}
        \item Saddle node bifurcation if you can create/destroy 2 fixed points by changing the parameter. Nullclines:
        \[\dot{x} = -ax + y,\quad and \quad \dot{y} = \frac{x^2}{1+x^2} - y\]
        \begin{align*}
            \eta_x &= x =\frac{y}a \implies y = ax \\
            \eta_y &= y = \frac{x^2}{1+x^2}
        \end{align*}
        Bifurcation:
        Hence fixed points if $x,y=0$ for all $a$, or 
        \begin{align*}            
        \frac{x^2}{1+x^2} - ax= 0\\
        x^2 - ax(1+x^2) = 0\\
        x - a + ax^2 = 0\\
        a = \frac{}{}
        \end{align*}
        \item Show pitchfork for:
        \[\dot x  =-bx + y + \sin x \quad \dot y = x-y\]
        \begin{align*}
           \eta_x = y &= bx-\sin x\\
           \eta_y = x&=y
        \end{align*}
        \begin{align*}
            x = bx - \sin x\\
            (1-b)x - \sin x = 0
        \end{align*}
        $x=0,y=0$ for all $b$ is a solution
    \end{enumerate}
    \item The ODE
    \[2tx^3 + 3t^2x^2 \frac{dx}{dt} = 0\]
    \begin{align*}
        2tx^3 + 3t^2x^2 \frac{dx}{dt} = 0\\
        2tx^3dt + 3t^2x^2 dx= 0\\
    \end{align*}
    Exact if it can be written as $f(x,t)dt + g(x,t)dx = 0$ its exact.
    \begin{align*}
        2tx^3 + 3t^2x^2 \frac{dx}{dt} = 0\\
        \frac{2}{3tx} + \frac{dx}{dt} = 0\\
    \end{align*}
    Hence linear
    \begin{align*}
        \frac2{3t} + x\frac{dx}{dt} =0
    \end{align*}
    Hence separable
    All of these show it is homogeneous.
    \item 
    \[f(x) = \log |x|\]
    This can't be globally Lipschitz continuous since it is not continuous about $x=0$. 
    Lipschitz continuous if
    \[|f(x) - f(y)| \leq L |x-y|\]
    hence
    \[\frac{df}{dx} = \begin{cases}
        \frac1x, &\text{ if }x > 0\\
        -\frac1x & \text{ if } x<0
    \end{cases}\]
    For $x,y > 0$
    \begin{align*}
        |\log|x| - \log|y|| &= |\log|\frac{x}{y}||\\
        &=||x|-|y|\frac1c| \quad (mvt)\\
        &= \frac1c |x-y|
    \end{align*}
    However $\frac1c$ is not bounded above. However for the intervals $(x,y) \in [-\infty,0)$ or $(x,y) \in (0,\infty]$

\end{enumerate}


\section{Tute 5}
\begin{enumerate}
    \item 
    \[\odd ut = u^2, \quad u(0)=1\]
    PL - if the interval $t_- \leq t_0 < t_+$ and $u_- < u_0 < u_+$ and $u^2$ is continuous in there, and Lipschitz continuous on the $u$ interval then it has a unique solution
    It is continuous.
    Lipschitz:

    \begin{align*}
        |x^2-y^2| = 2C|x-y|\\
    \end{align*}
    Hence by PL there exists a unique solution.
    Pick the interval with $t_0 = 0$ and $u_0 = 1$

    Pick $\alpha$ as the radius of temporal interval and $\delta$ as radius of spatial interval.

    Define 
    \[M = ||f|| = \sup_{I\times J} |f| = \sup_{J} |u^2| = (1+\delta)^2 \]
    \[\frac{\delta}{M} = \delta / (1+\delta)^2\]
    \[\epsilon = \min\{\alpha,\delta/M\} = \min\{\alpha, \delta/(1+\delta)^2\} < 1\]
    And
    \[\max\{\frac{\delta}{(1+\delta)^2}\} = \frac14, \quad \delta=1\]
    (apparently)


    \item 
    \[\odd ut = 1 + u^2 \ u(0) = 0\]
    \begin{align*}
        u_1 = 0 + \int_0^{t} 1 ds\\
        u_1 = t\\
        u_2 = 0 + \int_0^t 1 + s^2 ds\\
        u_2 = t + \frac{t^3}3\\
        u_n = \int_0^t 1+ u_{n-1}^2(s) ds
    \end{align*}

    \begin{verbatim}
    syms t
    u = 0 
    for k=1:10
       u = int(1+u^2,t,0,t);   
    end    
    \end{verbatim}
\item Finite diff formula
\[x_{n+1} = (2+h^2) x_n - x_{n-1} , \quad h=step\]
    \begin{enumerate}
        \item Which ODE does this pair with
        \begin{align*}
            x(t+h) = (2+h^2) x(t) - x(t-h)\\
            x(t+h) = 2x(t) + h^2x(t) - x(t-h)\\
            x(t) = \frac{x(t+h) - 2x(t) + x(t-h)}{h^2} \\
            x(t) = x''(t-h)\\
            x(t) = x''(t), \quad h\to 0 
        \end{align*}


        \item Order of local discretisation error
        \begin{align*}            
        \end{align*}
        \item Is this explicit or implicit
        \begin{align*}      
        \end{align*} 
    \end{enumerate}
\end{enumerate}

\section{Tute 6}
\end{document}
