\documentclass{E:/Documents/Latex/myassignment}
%%Document info
\title{Optimal Functions and Nanomechanics (OFN) TUTES}
\begin{document}

\section{Tute 1}
\begin{enumerate}
\item Revision
	\begin{enumerate}
			\item Degenerate cases when 
		\[y=x^n\]
		\item Geometrically it is the point between $\vec x$ and $\vec y$
		\item Partial derivative
		\[\dd f{x_i} = \lim_{h\to 0 } \frac{f(\vec x + \vec e_i h) - f(\vec x)}{h}\]
		\begin{align*}
			\dd{f(x,y)}{x} = 2x - 12x^3\\
			\dd{f(x,y)}{x} = \sin(2x+yz) + 2x\cos(2x+yz)\\
			\dd{f(x,y)}{x} = \ldots
		\end{align*}
		\item Gradients:
		\begin{align*}
			\Delta f(x,y) = \left(2x-12x^3y, 4y^3 - 3x^4\right)\\
			\ldots
		\end{align*}
		\item Taylor expansions
	\end{enumerate}
	\item Chain rule for $\odd zt$
	\begin{align*}
		\odd zt &= \dd zx \dd xt + \dd zy \dd yt
	\end{align*}
	\item Taylor's Theorem for a polynomial approximation for 
	\[f(x,y) = \sin(x+y^2)\]
	\[f(x+\delta x, y+ \delta y) = f(x,y) + \delta \vec x \Delta f(x,y) + \frac12 \delta \vec x^T H(x,y) \delta \vec x + \bigo(\delta \vec x ^3)\]
	\[\sin(x+\delta x+(y+\delta y)^2) = \sin(x+y^2) + \delta \vec x \begin{pmatrix}
		\cos(x+y^2)\\
		2y\sin(x+y^2)
	\end{pmatrix} + \frac12 \delta \vec x^T H(x,y) \delta \vec x + \bigo(\delta \vec x ^3)\]
	\item Cylinder of largest volume inside a unit sphere
	Volume of a cylinder 
	\[V = \pi r^2 * h\]
	Where $r$ is the radius of the cylinder
	\begin{align*}
		r^2 + \left(\frac{h}{2}\right)^2 \leq 1
	\end{align*}
	And $h,r > 0$.
	So we have
	\begin{align*}
		max \ V = \pi r^2h\\
		r^2 + \left(\frac{h}{2}\right)^2 -1 =0
	\end{align*}
	So we want
	\[H(r,h,\lambda) = \pi r^2 h + \lambda \left(r^2 + \left(\frac{h}{2}\right)^2 -1\right)\]
	Solve:
	\begin{align*}
	 \dd H r = 0 \\
	 \dd H h = 0\\
	 \dd H \lambda = 0	
	 \end{align*} ...

	 

	 \item 
\end{enumerate}
\section{Tute 2}
\begin{enumerate}
	\item 
	\begin{enumerate}
		\item 
		\begin{align*}
			F\{y\} &= \int_0^{\pi/2} (y^2 + y'^2 - 2y\sin x) dx 
		\end{align*}
		
		\begin{align*}
			\odd{}x\left(\dd f{y'}\right) - \dd f{y} =0\\
			\odd{}x(2y') - 2y + 2\sin x = 0\\
			2y'' - 2y + 2\sin x = 0\\
			y'' - y + \sin x = 0\\
		\end{align*}
		Homo:
		$y'' - y = 0$:
		\[y_h = a\sin x + b\cos x \]

		\item
		\[F\{y\} = \int_1^2 \frac{y'^2}{x^3} dx\] 
		No y dependence
		\begin{align*}
			\dd f{y'} = c_1^*\\
			\frac{2y'}{x^3} =c_1^*\\
			y' = c_1^+ x^3\\
			y = x^4 c_1 + c_2
		\end{align*}
		Use $y(1) = 0$ and $y(2) = 15$ but cbf
		\item
		\[F\{y\} = \int_0^2 (xy' + y'^2) dx \]
		No explicit y dependence vol 2
		\[\dd fy = 0\]
		\[\odd{}x \left(\dd f{y'}\right) + \dd fy = 0\]
		So
		\[\dd f{y'} =const \]
		\begin{align*}
			\dd f{y'} &= x + 2y' = const\\
			2y'&=c_1 -x\\
			y' &= \frac{c_1}2 -\frac{x}{2}\\
			y &= \frac{c_1 x}{2} - \frac{x^2}{4} + c_2
		\end{align*}
		$y(0) = 1 \implies c_2 =1$
		$y(2) = 0$
		\begin{align*}
			\frac{c_1 2}{2} - \frac{4}{4} +1 = 0\\
			c_1 = 0
		\end{align*}
	\end{enumerate}
	\item If you layer the shit out of glass then i guess so
	\item 
	Use conic coords
	\begin{align*}
		x = r\cos \theta \sin \alpha\\
		y = r\sin \theta \sin \alpha\\
		z = r\cos \alpha\\
	\end{align*}
	Short path
	\begin{align*}
	 	dx = \cos\theta \sin \alpha dr - r\sin\theta \sin \alpha d\theta\\
	 	dy = \sin\theta \sin \alpha dr + r\cos\theta \sin \alpha d\theta\\
	 	dz = \cos\alpha dr
	\end{align*} 
	\begin{align*}
		ds^2 &= dx^2 + dy^2 + dz^2\\
		&= \left(\cos\theta \sin \alpha dr - r\sin\theta \sin \alpha d\theta\right)^2 + \left(\sin\theta \sin \alpha dr + r\cos\theta \sin \alpha d\theta\right)^2 + \cos^2\alpha dr^2\\
		&= \cos^2 \theta \sin^2 \alpha dr^2 - 2r\sin\theta\cos\theta \sin^2 \alpha d\theta + r^2 \sin^2\theta \sin^2 \alpha d\theta^2 + \sin^2\theta \sin^2\alpha dr^2 \\&+ 2r \sin\theta\cos\theta\sin^2 \alpha d\theta + r^2\cos^2\theta \sin^2 \alpha d\theta^2 + \cos^2 \alpha dr^2\\
		&= \sin^2\alpha dr^2 + r^2 \sin^2\alpha d\theta^2 + \cos^2 \alpha dr^2\\
		&= dr^2 + r^2 \sin^2\alpha d\theta^2
	\end{align*}

	\begin{align*}
		F\{y\} &= \int_{P_1}^{P_2} ds\\
		&= \int_{P_1}^{P_2} \sqrt{dr^2 + r^2 \sin^2\alpha d\theta^2}\\
		&= \int_{P_1}^{P_2} \sqrt{1 + r^2 \sin^2\alpha \left(\frac{d\theta}{dr}\right)^{2}} dr\\
	\end{align*}
	So
	\[f(r,\theta,\theta') = 1 + r^2 \sin^2 \alpha \theta'^2\]
	No $\theta$ dependence.
	\[\dd f{\theta'} =  c_1\]
	\item 
	\item 
\end{enumerate}

\section{Tutorial 3}
\begin{enumerate}
	\item 
	\[F\{y\} = \int f(x,y,y',y'',y''') dx\]
	Taylor's theorem
	\begin{align*}
		&f(x,y+\epsilon \eta, y' + \epsilon'\eta', y'' + \epsilon\eta '', y'''+ \epsilon\eta''')\\
		&= f(x,y,y',y'',y''') + \epsilon(\eta \dd fy + \eta' \dd f{y'} + \eta'' \dd f{y''} + \eta''' \dd f{y'''}) + \bigo(\epsilon^2)
	\end{align*}
	\[F\{y+\epsilon\eta\} = \int_{x_0}^{x_1} f(x,y,y',y'',y''') + \epsilon(\eta \dd fy + \eta' \dd f{y'} + \eta'' \dd f{y''} + \eta''' \dd f{y'''}) + \bigo(\epsilon^2) dx\]
	\begin{align*}
		\delta F &= \lim_{\epsilon \to 0} \frac{F\{y + \epsilon \eta\} - F\{y\}}{\epsilon}\\
		&=\lim_{\epsilon \to 0} \int_{x_0}^{x_1} f(x,y,y',y'',y''')/\epsilon + (\eta \dd fy + \eta' \dd f{y'} + \eta'' \dd f{y''} + \eta''' \dd f{y'''}) + \bigo(\epsilon) - f(x,y,y',y'',y''')/\epsilon\\
		&=\lim_{\epsilon \to 0} \int_{x_0}^{x_1}  (\eta \dd fy + \eta' \dd f{y'} + \eta'' \dd f{y''} + \eta''' \dd f{y'''}) + \bigo(\epsilon)\\
	\end{align*}
	Integrate by parts
	\item Geodesics in $N$ dim Euclidean space, assume $\mathbb{R}^N$ with $\vec q= (q_1,\ldots,q_n)$ with $||\vec q|| = \left(\sum_{n=1}^N q_n^2\right)^{1/2}$ find the extremal of
	\[S\{\vec q(t)\} = \int ds\]
	\item Assuming every atom of a carbon nanotorus with genus $g=1$ is bonded to exactly 3 neighbours, how many pentagonal, hexagonal and heptagonal rings must occur when assuming that
	\begin{enumerate}
		\item 
		\item 
		\item 
	\end{enumerate}
	\item For some $n\in \mathbb{N}$ show
	\[\Gamma(n+1/2) = \frac{\sqrt{\pi} (2n-1)!!}{2^n}\]
	Where $!!$ is the double factorial $(n!! = n(n-2)(n-4)\ldots)$
	\item 
	\[B(x,y)B(x+y,z) = B(y,z) B(y+z,x) = B(z,x) B(z+x,y)\]
	$B(x,y) = \frac{\Gamma(x) \Gamma(y)}{\Gamma(x+y)}$
	\begin{align*}
		B(x,y)B(x+y,z) &= \frac{\Gamma(x) \Gamma(y)}{\Gamma(x+y)} \frac{\Gamma(x+y) \Gamma(z)}{\Gamma(x+y+z)}\\
		&= \frac{\Gamma(x) \Gamma(y)\Gamma(z)}{\Gamma(x+y+z)}\\
		B(y,z) B(y+z,x) &= \frac{\Gamma(x) \Gamma(y)\Gamma(z)}{\Gamma(x+y+z)}\\
		B(z,x) B(z+x,y) &= \frac{\Gamma(x) \Gamma(y)\Gamma(z)}{\Gamma(x+y+z)}\\
	\end{align*}

	\item 
	\[F(a,b;c;z) = \sum_{n=0}^\infty \frac{(a)_n(b)_n}{(c)_n n!} z^n\]
	\begin{enumerate}
		\item 
		\[\odd{}z F(a,b;c;z) = \frac{ab}{c} F(a+1,b+1;c+1;z)\]
		\begin{align*}
			\odd{}z F(a,b;c;z) &= \odd{}z\sum_{n=0}^\infty \frac{(a)_n(b)_n}{(c)_n n!} z^n\\
			&=\sum_{n=0}^\infty \frac{(a)_n(b)_n}{(c)_n n!} nz^{n-1}\\
			&= \sum_{n=0}^\infty \frac{(a)_{n+1}(b)_{n+1}}{(c)_{n+1} (n+1)!} (n+1)z^{n}\\
		\end{align*}
		Note 
		\[(a)_{n+1} = \frac{\Gamma(a+n+1)}{\Gamma(a)} = \frac{a\Gamma(a+n)}{\Gamma(a)} = a(a)_{n}\]
		\begin{align*}
			\odd{}z F(a,b;c;z) &= \sum_{n=0}^\infty \frac{(a)_{n+1}(b)_{n+1}}{(c)_{n+1} (n+1)!} (n+1)z^{n}\\
			\odd{}z F(a,b;c;z) &= \frac{ab}{c}\sum_{n=0}^\infty \frac{(a)_{n}(b)_{n}}{(c)_{n} (n+1)!} (n+1)z^{n}\\
			\odd{}z F(a,b;c;z) &= \frac{ab}{c}\sum_{n=0}^\infty \frac{(a)_{n}(b)_{n}}{(c)_{n} n!} z^{n}\\
			&= \frac{ab}{c} F(a+1,b+1;c+1;z)
		\end{align*}
		\item 
		\[\left[\odd{}z F(a,b;c;z)\right]_{z=0} = \frac{ab}{c}\]
		The only non-zero term is $z^0 = 0^0 = 0$ and hence you get $\frac{ab}{c}$ 
		\item 
		\[\frac{1}{\sqrt{1-z}} = F(1/2,b;b;z)\]
		\begin{align*}
			F(1/2,b;b;z) &= \sum_{n=0}^\infty \frac{(1/2)_n(b)_n}{(b)_n n!} z^n\\
			&= \sum_{n=0}^\infty \frac{(1/2)_n}{n!} z^n\\
			&= \sum_{n=0}^\infty \frac{\Gamma(1/2 +n)}{\Gamma(1/2) n!} z^n\\
			&= \sum_{n=0}^\infty \frac{\Gamma(1/2 +n)}{\Gamma(1/2) \Gamma(n+1)} z^n\\
			&= \sum_{n=0}^\infty \ncr{1/2}{n} z^n\\
			&= \frac{1}{\sqrt{1-z}}
		\end{align*}
		\[\frac{1}{\sqrt{1-z}} = \sum_{n=0}^\infty z^n \ncr{1/2}{n}\]
		Note
		\[\ncr{x}{y} =\frac{x!}{y!(x-y)!} = \frac{\Gamma(x+1)}{\Gamma(y+1) \Gamma(x-y+1)}\]
		Assuming int at first then for all numbers.
	\end{enumerate}
\end{enumerate}
\section{Tute 4}
\begin{enumerate}
	\item Assume $\Phi(\rho) = -A \rho^{-6} + B \rho^{-12}$ calculate the interaction of
	\begin{enumerate}
		\item A point $P = (0,0,\delta)$ and a line $\ell_1$ collinear with the $x-axis$, $-\infty < x < \infty$, with uniform line density $\eta_1$.

		$\ell_1 = (x,0,0)$
		The distance from the point to a point on the line is 
		\[d = \sqrt{x^2 + \delta^2}\]

		\begin{align*}
			E &= \eta_1\eta_1 \int_{S_2} \int_{S_1} \Phi(\rho) dA_1 dA_2\\
			&=
		\end{align*}

		\item $\ell_1$ from $A$ and $\ell_2$ parameterised by $(t\cos\theta, t\sin\theta, \delta)$
		\item Whats the interaction between $\ell_1$ and $\ell_2$ for $\theta = \pi/2$
		\item Same as before but $\theta = \pi$.
	\end{enumerate}
	\item The surface $\mathcal{T}$:
	\[\vec r(\theta,\phi) = \left((R + r\cos\theta) \cos\phi, (R+r\cos\theta) \sin\phi,r\sin\theta\right)\]
	With radius $r$ and $R$ distance from the centre of the torus to the centre of the tube, $-\pi < \theta \leq \pi, \ -\pi < \phi \leq \pi$.
	\begin{enumerate}
		\item tangent vector in $\theta$ direction:
		\[\dd{r}{\theta} = \begin{pmatrix}
			-r\sin\theta\cos\phi\\-r\sin\theta\sin\phi\\ r\cos\theta
		\end{pmatrix}\]		
		\item tangent vector in $\phi$ direction
		\[\dd{r}{\phi} = \begin{pmatrix}
			-(R+r\cos\theta)\sin\phi \\ (R+r\cos\theta)\cos\phi\\0
		\end{pmatrix}\]
		\item Determine $dA$ for $\mathcal{T}$.
		\[	dA = \dd r\theta \times \dd r\phi d\theta d\phi\]
		\[dA = \begin{vmatrix}
			\vec i & \vec j &\vec k \\
			-r\sin\theta\cos\phi&-r\sin\theta\sin\phi& r\cos\theta\\
			-(R+r\cos\theta)\sin\phi & (R+r\cos\theta)\cos\phi&0
		\end{vmatrix}d\theta d\phi\]
		\[dA = \begin{pmatrix}
			0 - r\cos\theta(R+r\cos\theta)\cos\phi\\
			-r\cos\theta (R+r\cos\theta)\sin\phi\\
			-r\sin\theta\cos\phi (R+r\cos\theta)\cos\phi - r(R+r\cos\theta)\sin\phi \sin\theta\sin\phi
		\end{pmatrix}d\theta d\phi\]
		\[dA = - r(R+r\cos\theta)\begin{pmatrix}
			\cos\theta\cos\phi\\
			\cos\theta\sin\phi\\
			\sin\theta
		\end{pmatrix}d\theta d\phi\]
		\item Derive an expression for the SA of $\mathcal{T}$.
		\begin{align*}
			SA &= \int_{-\pi}^\pi \int_{-\pi}^\pi dA\\
		\end{align*}
	\end{enumerate}
	\item Explicitly told to avoid this one
	\item Use Ritz's method to minimise
	\[J\{y\} = \int_0^{2\pi} (y'^2 + \lambda^2 y^2) dx\]
	With $y(0) = 1$ and $y(2\pi) = 1$ and $\lambda$ is positive integer. Use
	\[\phi_0 = 1, \quad \phi_n(x) = \sin[(n-1/2)x]\]
	And
	\[y_N = \phi_0 + \sum_{n=1}^N c_n \phi_n(x)\]

	\[y_n' =  c_1(n-1/2)\cos[(n-1/2)x]\]
	\begin{align*}
		J_1(c_1) &= \int_0^{2\pi} (y'^2 + \lambda^2 y^2) dx\\
		&= \int_0^{2\pi} (c_1(n-1/2)\cos[(n-1/2)x])^2 + \lambda^2(c_1\sin[(n-1/2)x])^2 dx\\
		&= c_1^2\int_0^{2\pi} ((n-1/2)\cos[(n-1/2)x])^2 + \lambda^2(\sin[(n-1/2)x])^2 dx\\
		&= c_1^2\int_0^{2\pi} ((n-1/2)^2-\lambda^2)\cos^2[(n-1/2)x] + \lambda^2 dx\\
		&= c_1^2 \left(((n-1/2)^2-\lambda^2)\int_0^{2\pi} \cos^2[(n-1/2)x] dx + \int_0^{2\pi} \lambda^2 dx\right) \\
		&= c_1^2 \left(\frac{((n-1/2)^2-\lambda^2)(\sin(2 (n-1/2)x) + 2(n-1/2)x)}{4(n-1/2)}\right)_0^{2\pi} + 2\pi\lambda^2 \\
		&= c_1^2 \pi + 2\pi\lambda^2 \\
	\end{align*}
\end{enumerate}



\section{Tute 4}
\begin{enumerate}
	\item Consider the cat problem again but with
	\[W_p\{y\} = g\int_0^1 c_1 + c_2 y dx\]
	Include the isoperimetric constraint to get
	\[W_p\{y\} = \int_0^1 my + \lambda \sqrt{1+y'^2}dx\]
	Solve for the shape of the cable
	\[\dd fy = my', \quad \dd f{y'} = \frac{\lambda y'}{\sqrt{1+y'^2}}\]
	\begin{align*}
	\frac{\lambda y'^2}{\sqrt{1+y'^2}} - (my + \lambda \sqrt{1+y'^2}) = c\\
	\lambda y'^2 - (my\sqrt{1+y'^2} + \lambda (1+y'^2)^2) = c\sqrt{1+y'^2}\\
	\lambda y'^2 - (my\sqrt{1+y'^2} + \lambda (1+2y'^2 + y'^4)) = c\sqrt{1+y'^2}\\
	\lambda = \sqrt{1+y'^2} (c + my)
	\end{align*}




	\item Two boats one towing the other. Only the first boat has a motor, and the second boat only uses its rudder to maintain a constant horizontal distance $=1$.

	The force on a section of the tow rope is proportional to the length of that section. 
	\[\]

	\item
	\item 
	\[F\{y\} = \int_0^R \frac{x}{1+y'^2} dx\]
	With $y(0) = L$ and $y(R) = 0$ ($y' \leq 0$ and $y'' \geq 0$)
	Approximate $1+y'^2 \approx y'^2$
	\begin{enumerate}
		\item Solve it given the total surface area
		\item Find the optimal profile of the nose-cone.
		
	\end{enumerate}
\end{enumerate}


\section{Tute 6}
\begin{enumerate}
	\item Note the derivative of $\arctan(y)$ is $\frac{1}{1 + y'^2 }$
	\[f = K(x,y) e^{\arctan y'} \sqrt{1+y'^2}\]
	\begin{align*}
		p &= \dd f{y'}\\
		&= K\left(e^{\arctan y'} \frac{y'}{\sqrt{1 + y'^2}} + \frac{1}{1 + y'^2} e^{\arctan y'} \sqrt{1+ y'^2}\right)\\
		&= K e^{\arctan y'} \left(\frac{y' + 1}{\sqrt{1 + y'^2}}\right)
	\end{align*}
	\begin{align*}
		H &= y' \dd f{y'} - f\\
		&=K e^{\arctan y'} \left(\frac{y'^2 + y'}{\sqrt{1 + y'^2}}\right) - K(x,y) e^{\arctan y'} \sqrt{1+y'^2}\\
		&= \frac{Ke^{\arctan y'} (y' - 1)}{\sqrt{1+y'^2}}
	\end{align*}
	Hence the transversality condition is
	\begin{align*}
		\left[p \dd{y_{\Gamma}}{\xi} - H \dd{x_\Gamma}{\xi}\right]_{x_1} &= 0
		\frac{Ke^{\arctan y'}}{\sqrt{1+y'^2}}\left(y ' +1,y' -1\right) \cdot \left(1, \phi'\right) &=0\\
		\left(y ' +1,y' -1\right) \cdot \left(1, \phi'\right) &=0\\
		\left(\cos\pi/4 - \sin\pi/4 y', \cos \pi/4 + \sin \pi/4 y'\right) \cdot (1,\phi') &= 0 
	\end{align*}
	The last step is cheating.

	I.e. the extremal will approach at angle $\pi/4$.

	\item Distance from the point $(1,1,1)$ to
	\[x^2 + y^2 + z^2 = 1\]
	Using CoV

	\[F(y) =  \]


	\item Broken extremal
	\begin{align*}
		 	
	\end{align*} 
	\item 
\end{enumerate}


\end{document}